\documentclass{article}
\author{Miguel Cid Flor}
\usepackage[utf8]{inputenc}
\usepackage{amsmath, amssymb}

\title{C3 linearization properties}
\begin{document}

\maketitle
\section*{Notation Convention}

We adopt the following convention throughout the article:

\begin{itemize}
    \item Variables with a tilde (e.g.,  $\tilde{C}$) represent \textbf{sequences} (finite ordered lists of elements).
    \item Variables with a double tilde (e.g.,  $\tilde{\tilde{C}}$) represent \textbf{sequence of sequences}.
\end{itemize}

\section*{Ingredients}

$\mathcal{C}$ of classes $C_0, C_1, C_2, \ldots, C_n$.\\
Set $\mathcal{D}$ of pairs (class, set of classes) $\mathcal{C} \times \mathcal{\tilde{D}}$.\\
$\text{MRO}_D : \mathcal{C} \Rightarrow \mathcal{\tilde{C}}$.



\section*{Remove}
$remove : \mathcal{\tilde{\tilde{C}}} \times C \Rightarrow \mathcal{\tilde{\tilde{C}}}$\\
Let $\tilde{\tilde{L}} = [\tilde{L}_1; \ldots ; \tilde{L}_n]$, $\tilde{\tilde{L}} \in \mathcal{\tilde{\tilde{C}}}$\\
Let $C \in \mathcal{C}$\\

\[
\frac{
  C \quad \tilde{\tilde{L}} = [\tilde{L}_1 \setminus \{C\}, \dots, \tilde{L}_n \setminus \{C\}]
}{
  \text{remove}(\tilde{\tilde{L}}, C) = \tilde{\tilde{L}}
}
\]

\vspace{2cm}
\section*{Merge}
$merge : \mathcal{\tilde{\tilde{C}}} \Rightarrow \tilde{\mathcal{C}} $ \\
Let $\tilde{\tilde{L}} = [\tilde{L}_1; \ldots ; \tilde{L}_n]$, $\tilde{\tilde{L}} \in \mathcal{\tilde{\tilde{C}}}$\\


\[
merge(\tilde{\tilde{L}}) =
\begin{cases}
[C] \cdot merge(remove(\tilde{\tilde{L}}, C)), & \text{if } (\exists k \in \{1, \ldots, n\}, \tilde{L}_k \neq \emptyset \land C = head(\tilde{L}_k)) \land \\
& (\forall j < k, C \neq head(\tilde{L}_j)) \land \\ 
& (\forall i \in \{1,\ldots,n\},C \notin tail(\tilde{L}_i)) \\ 

fail  & otherwise
\end{cases}
\]

\vspace{2cm}
\section*{Extract Classes}
$classes : \tilde{\mathcal{D}} \Rightarrow \tilde{\mathcal{C}}$\\
Let $\tilde{D} = [(C_1, \tilde{P}_1), \ldots, (C_n, \tilde{P}_n)] \in \mathcal{\tilde{D}}$ \\

\[
classes(\tilde{D}) = [C_1, \ldots, C_n]
\]
\vspace{2cm}
\section*{C3 Linearization}
$c3linearization : \mathcal{D} \Rightarrow \tilde{\mathcal{C}}$\\
$\text{Let } D = (C, \tilde{P}) \text{ where } D \in \mathcal{D}$\\
\[
c3linearization(D) =
\begin{cases}
[C] & \text{if } \tilde{P} = \emptyset \\
\begin{split}
[C] \cdot merge([&c3linearization(P_1), \ldots, \\
         &c3linearization(P_n), \\
         &classes(\tilde{P})]) 
\end{split} & \text{otherwise}
\end{cases}
\]




\end{document}

